%%%%%%%%%%%%%%%%%%%%%%%%%%%%%%%%%%%%%%%
% Leik Lima-Eriksen - Resume
% Created 1/4/2019
% 
% Based on the highly popular resume 
% template made by Debarghya Das.
%%%%%%%%%%%%%%%%%%%%%%%%%%%%%%%%%%%%%%%

\documentclass[]{resume-build}


\begin{document}

%%%%%%%%%%%%%%%%%%%%%%%%%%%%%%%%%%%%%%
%
%     TITLE NAME
%
%%%%%%%%%%%%%%%%%%%%%%%%%%%%%%%%%%%%%%

\namesection{Leik Lima-Eriksen}
{images/profile_picture.png}
{\urlstyle{same}
	\faEnvelope \href{mailto:leik.lima-eriksen@lyse.net}{\hspace{0.5em}leik.lima-eriksen@lyse.net}\\
	\faGithub \href{https://github.com/leiklier}{\hspace{0.5em}github.com/leiklier}\\
	\faLinkedin \href{https://www.linkedin.com/in/leiklimaeriksen}{\hspace{0.5em}linkedin.com/in/leiklimaeriksen}
}
    
%%%%%%%%%%%%%%%%%%%%%%%%%%%%%%%%%%%%%%
%
%     COLUMN ONE
%
%%%%%%%%%%%%%%%%%%%%%%%%%%%%%%%%%%%%%%
\begin{minipage}[t]{0.34\textwidth} 
  
%%%%%%%%%%%%%%%%%%%%%%%%%%%%%%%%%%%%%%
%     EDUCATION
%%%%%%%%%%%%%%%%%%%%%%%%%%%%%%%%%%%%%%
\section{Education} 

\subsection{NTNU Gløshaugen}
\descript{Autumn 2017 - present}
\href{https://www.ntnu.no/studier/mtelsys}{MTech. in Electronics, 2. year} \\
Graduating June 2022 \\
Average grade: \emphasize{5.22} / 6 \\
\sectionsep

\subsection{St. Olav VGS, Stavanger}
\descript{Autumn 2014 - Spring 2017}
Programme subjects: IT 1 \& 2, Mathematics R1 \& R2, Physics 1 \& 2, and Chemistry 1 \\
Average grade: \emphasize{5.44} / 6 \\
\sectionsep

%%%%%%%%%%%%%%%%%%%%%%%%%%%%%%%%%%%%%%
%     SKILLS
%%%%%%%%%%%%%%%%%%%%%%%%%%%%%%%%%%%%%%
\section{Skills}
\subsection{Programming Languages:}
Javascript ES6, Python 3, C++11, HTML, CSS, LaTeX, and some PHP
\subsectionsep

\subsection{Frameworks:}
NodeJS, React, Ant Design, Redux, ExpressJS, Chart.js, and Apache
\subsectionsep

\subsection{Database Systems:}
MongoDB, and MySQL
\subsectionsep

\subsection{Tools:}
Git, vim, SSH, tmux, and Altium 19
\sectionsep
%
%
%%%%%%%%%%%%%%%%%%%%%%%%%%%%%%%%%%%%%%
%     COURSEWORK
%%%%%%%%%%%%%%%%%%%%%%%%%%%%%%%%%%%%%%
\section{Coursework}
\subsectionsep
\begin{tightemize}
\item Procedural- and object oriented programming (C++)
\item Information Technology (Python)
\item Statistics
\item Calculus 1 \&  2
\item Complex Calculus (Mathematics 4K)
\item Linear algebra
\item Electrical circuits and digital design
\item Computers and digital design
\end{tightemize}
\sectionsep

\DTMsetdatestyle{mylastupdate}
Last updated \DTMdisplaydate{\the\year}{\the\month}{\the\day}{-1}

%%%%%%%%%%%%%%%%%%%%%%%%%%%%%%%%%%%%%%
%
%     COLUMN TWO
%
%%%%%%%%%%%%%%%%%%%%%%%%%%%%%%%%%%%%%%
\end{minipage} 
\hfill
\begin{minipage}[t]{0.65\textwidth} 

%%%%%%%%%%%%%%%%%%%%%%%%%%%%%%%%%%%%%%
%     ADDITIONAL INFORMATION
%%%%%%%%%%%%%%%%%%%%%%%%%%%%%%%%%%%%%%
\section{Recreational activities}

\activity{Orbit NTNU}{Jan 2019 - present}\\
\position{Satellite Communications Engineer}{ }
\sectionsep
\begin{tightemize}
    \item Configuring the ground station responsible for the communication with a cube satellite developed by students.
\end{tightemize}

\activity{Omega Verksted NTNU}{Feb 2019 - present}\\
%\position{Member}{ }
\begin{tightemize}
    \item A lot of geeking, which includes PCB design and microcontroller programming.
\end{tightemize}

\sectionsep


%%%%%%%%%%%%%%%%%%%%%%%%%%%%%%%%%%%%%%
%     EXPERIENCE
%%%%%%%%%%%%%%%%%%%%%%%%%%%%%%%%%%%%%%
\section{Work experience}

\workplace{Volunteering work for Follow Me}{2013 – 2017}\\
\position{Sound engineer}{Stavanger, Rogaland}
\begin{tightemize}
\item Responsible for rigging up and down sound- and light equipments for multiple small concerts in Stavanger.
\item Mixed sound at the above events and have been responsible for audio recording.
\end{tightemize}
\sectionsep

\workplace{Part-time job in an amusement park}{Apr – Jul 2016} \\
\position{Responsible for the attractions}{Ålgård, Rogaland}
\begin{tightemize}
\item Informed the guest on how they should behave in the attractions, and was responsible for controlling these.
\end{tightemize}
\sectionsep

%%%%%%%%%%%%%%%%%%%%%%%%%%%%%%%%%%%%%%
%     PROJECTS
%%%%%%%%%%%%%%%%%%%%%%%%%%%%%%%%%%%%%%
\section{Projects}
\project{Website for IoT devices}{leiklier/smartwater-website}\\
\descript{}
A web application which visualizes data sent from sensor nodes through both line diagrams and tables. Receives data in real-time via WebSockets. \emphasize{React} application with \emphasize{Redux}, \emphasize{Ant Design}, and \emphasize{Chart.js}.
\subsectionsep

\project{API for IoT devices}{leiklier/smartwater-api}\\
\location{Chosen to be used by 7 out of 8 groups participating in the subject}
\descript{}
RESTful CRUDL-type HTTP-exposed API which stores data sent from The Things Network in a database for later retrieveal. Real-time updates through WebSockets. Has successfully received and stored 170 000 measurements without ever crashing. Build with \emphasize{ExpressJS}, \emphasize{MongoDB}, and \emphasize{Websockets}.
\subsectionsep 

\project{PCB design of an IoT device}{leiklier/smartwater-pcb}\\
\descript{}
Designed the PCB for an IoT device which measures diverse parameters and transfers these over LoRaWAN. The system is self-supplied with energy through a solar panel and rechargeable batteries. Design done in \emphasize{Altium 19}.
\subsectionsep


\end{minipage} 

\end{document}  
